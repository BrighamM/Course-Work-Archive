


\documentclass[usepdftitle=false,professionalfonts,compress ]{beamer}

%Packages to be included
\usepackage[latin1]{inputenc}
\usepackage{graphics,epsfig, subfigure}
\usepackage{url}
\usepackage[T1]{fontenc}
%\usepackage{listings}
\usepackage{hyperref}
\usepackage[english]{babel}

%%%%%%%%%%%%%%%%%%%%%%%%%%%%%%%%%%%%%%%%%%%%%%%%%
%%%%%%%%%% PDF meta data inserted here %%%%%%%%%%
%%%%%%%%%%%%%%%%%%%%%%%%%%%%%%%%%%%%%%%%%%%%%%%%%
\hypersetup{
	pdftitle={02-MIPS},
	pdfauthor={}
}





%%%%%% Beamer Theme %%%%%%%%%%%%%

\usetheme[]{Warsaw}
		
\title{02-MIPS}






%%%%%%%%%%%%%%%%%%%%%%%%%%%%%%%%%%%%%%%%%%%%%%%%%
%%%%%%%%%% Begin Document  %%%%%%%%%%%%%%%%%%%%%%
%%%%%%%%%%%%%%%%%%%%%%%%%%%%%%%%%%%%%%%%%%%%%%%%%




\begin{document}
\frame[plain]{
	\frametitle{}
	\titlepage
	\vspace{-0.5cm}
	\begin{center}
	%\frontpagelogo
	\end{center}
}
\frame{
	\tableofcontents[hideallsubsections]
}






%%%%%%%%%%%%%%%%%%%%%%%%%%%%%%%%%%%%%%%%%   
%%%%%%%%%% Content starts here %%%%%%%%%%
%%%%%%%%%%%%%%%%%%%%%%%%%%%%%%%%%%%%%%%%%







\section{MIPS}
		
\subsection{Overview}

{
\begin{frame}\frametitle{MIPS}

	\begin{itemize}
	\item Microprocessor without Interlocked Pipeline Stages
			\item Developed in 1981 by John L. Hennessy
			\item One of the first RISC archetectures
			\item Common in embedded systems
				\end{itemize}

\end{frame}}






\subsection{Instructions}

{
\begin{frame}\frametitle{Signed v. Unsigned}

	\begin{itemize}
	\item Signed integers are stored as 2's complement
			\item Unsigned integers are stored as binary
			\item The different formats can not be

	\begin{itemize}
	\item compared
			\item used in the same expression
			\item safely converted
				\end{itemize}

				\end{itemize}

\end{frame}}





{
\begin{frame}\frametitle{Co-processor}

	\begin{itemize}
	\item Floating point instructions on floating point co-processor
			\item Co-processor has a separate set of instructions
			\item For simplicity CPSL does not have floating point support
				\end{itemize}

\end{frame}}





{
\begin{frame}\frametitle{Load / Store}

	\begin{itemize}
	\item li
			\item la
			\item lw
			\item sw
				\end{itemize}

\end{frame}}






{
\begin{frame}\frametitle{Arithmetic}

	\begin{itemize}
	\item add, addi, addu, addiu
			\item sub, subu
			\item mult, multu
			\item div, divu
			\item mfhi, mflo -- Used to get results of mult and div
				\end{itemize}

\end{frame}}







{
\begin{frame}\frametitle{Conditionals}

	\begin{itemize}
	\item slt, sltu, slti, sltiu
			\item and, andi
			\item or, ori
			\item nor
				\end{itemize}

\end{frame}}






{
\begin{frame}\frametitle{Branching}

	\begin{itemize}
	\item beq, bne
			\item j, jr, jal
				\end{itemize}

\end{frame}}




\subsection{Registers}

{
\begin{frame}\frametitle{Types}

	\begin{itemize}
	\item 32 registers \$0 - \$31
			\item mnemonic names to help remember usage
				\end{itemize}

\end{frame}}




{
\begin{frame}\frametitle{Reserved}

	\begin{itemize}
	\item \$zero - Always 0
			\item \$at - Assembler Temporary
			\item \$k0, \$k1 - Reserved for Kernal
				\end{itemize}

\end{frame}}





{
\begin{frame}\frametitle{Variables}

	\begin{itemize}
	\item \$s0-\$s7 - Saved Registers
			\item \$t0-\$t9 - Temporary Registers
				\end{itemize}

\end{frame}}




{
\begin{frame}\frametitle{Function Calls}

	\begin{itemize}
	\item \$v0, \$v1 - Return values
			\item \$a0-\$a3 - Arguments
			\item \$ra - return address - set by jal
				\end{itemize}

\end{frame}}





{
\begin{frame}\frametitle{Memory Management}

	\begin{itemize}
	\item \$gp - Global pointer
			\item \$sp - Stack pointer
			\item \$fp - Frame pointer
				\end{itemize}

\end{frame}}





{
\begin{frame}\frametitle{Conventions}

	\begin{itemize}
	\item The following registers must have the same value after a function call returns as they had before the function call:

	\begin{itemize}
	\item Saved registers
			\item Global pointer
			\item Stack pointer
			\item Frame pointer
			\item Return address
				\end{itemize}

			\item All other registers are in an unknown state after a function call
				\end{itemize}

\end{frame}}




\subsection{System Calls}

{
\begin{frame}\frametitle{syscall}

	\begin{itemize}
	\item I/O provided by simulator
			\item put system call number into \$v0
			\item 1 -- print integer -- \$a0 contains integer to print
			\item 4 -- print string -- \$a0 contains address of string
			\item 5 -- read integer -- \$v0 contains integer value
			\item 8 -- read string -- \$a0 contains buffer, \$a1 contains length
			\item 10 -- exit
				\end{itemize}

\end{frame}}










\section{Exercises}
		
\subsection{Expression}

{
\begin{frame}\frametitle{$Distance^2$}

\end{frame}}

\subsection{Loop}

{
\begin{frame}\frametitle{Sum of an Array}

\end{frame}}

\subsection{Function Call}

{
\begin{frame}\frametitle{Recursive Factorial}

\end{frame}}





\end{document}