


\documentclass[usepdftitle=false,professionalfonts,compress ]{beamer}

%Packages to be included
\usepackage[latin1]{inputenc}
\usepackage{graphics,epsfig, subfigure}
\usepackage{url}
\usepackage[T1]{fontenc}
%\usepackage{listings}
\usepackage{hyperref}
\usepackage[english]{babel}

%%%%%%%%%%%%%%%%%%%%%%%%%%%%%%%%%%%%%%%%%%%%%%%%%
%%%%%%%%%% PDF meta data inserted here %%%%%%%%%%
%%%%%%%%%%%%%%%%%%%%%%%%%%%%%%%%%%%%%%%%%%%%%%%%%
\hypersetup{
	pdftitle={Lexical Analysis},
	pdfauthor={KennethSundberg}
}





%%%%%% Beamer Theme %%%%%%%%%%%%%

\usetheme[]{Warsaw}
		
\title{Lexical Analysis}
\subtitle{CS 5300}
\author{KennethSundberg}
\date{}






%%%%%%%%%%%%%%%%%%%%%%%%%%%%%%%%%%%%%%%%%%%%%%%%%
%%%%%%%%%% Begin Document  %%%%%%%%%%%%%%%%%%%%%%
%%%%%%%%%%%%%%%%%%%%%%%%%%%%%%%%%%%%%%%%%%%%%%%%%




\begin{document}
\frame[plain]{
	\frametitle{}
	\titlepage
	\vspace{-0.5cm}
	\begin{center}
	%\frontpagelogo
	\end{center}
}
\frame{
	\tableofcontents[hideallsubsections]
}






%%%%%%%%%%%%%%%%%%%%%%%%%%%%%%%%%%%%%%%%%   
%%%%%%%%%% Content starts here %%%%%%%%%%
%%%%%%%%%%%%%%%%%%%%%%%%%%%%%%%%%%%%%%%%%










\section{Role}
		
\subsection{Overview}

{
\begin{frame}\frametitle{Features of Lexical Analyzer}

	\begin{itemize}
	\item Groups characters into tokens for syntax analyzer
			\item Strips white space
			\item Keeps track of line numbers (for diagnostics)
			\item Generates output listing with errors marked
			\item Delete comments
			\item Converts numbers to binary form
				\end{itemize}

\end{frame}}








{
\begin{frame}\frametitle{Microsyntax}

	\begin{itemize}
	\item Scanner recognizes rule for the words of the language
			\item Type of word -- Part of speach can be determined by scanner
				\end{itemize}

\end{frame}}




{
\begin{frame}\frametitle{Input}

	\begin{itemize}
	\item Sequence of characters
			\item Lexer takes file of characters produces tokens
			\item Parser calls scanner which returns next token
				\end{itemize}

\end{frame}}





{
\begin{frame}\frametitle{Output}

	\begin{itemize}
	\item Sequence of tokens

	\begin{itemize}
	\item Operators
			\item puctuation
			\item keywords
			\item identifiers
			\item string literals
			\item character literals
			\item numeric literals

	\begin{itemize}
	\item base 10
			\item octal
			\item hex
			\item floating point
				\end{itemize}

				\end{itemize}

				\end{itemize}

\end{frame}}



{
\begin{frame}\frametitle{Terminology}

	\begin{itemize}
	\item Token

	\begin{itemize}
	\item Type of word
			\item Part of Speech
			\item Implemented as enumeration
				\end{itemize}

			\item Lexeme

	\begin{itemize}
	\item Some Kinds of token need associated data
			\item Value derived from matched text
				\end{itemize}

			\item Pattern

	\begin{itemize}
	\item Regular expressions used to recognize tokens
				\end{itemize}

				\end{itemize}

\end{frame}}





{
\begin{frame}\frametitle{Recognizing Words}

	\begin{itemize}
	\item Part of Speech
				\end{itemize}

\end{frame}}



{
\begin{frame}\frametitle{Keywords}

	\begin{itemize}
	\item Reserved identifiers
			\item Only token kind is significant
				\end{itemize}

\end{frame}}




{
\begin{frame}\frametitle{Literals}

	\begin{itemize}
	\item String, character, or numeric
			\item Both token kind and lexem are significant
			\item lexeme may be interpreted first
				\end{itemize}

\end{frame}}





{
\begin{frame}\frametitle{Operators}

	\begin{itemize}
	\item Like separators
			\item No real difference for lexical analyzer
				\end{itemize}

\end{frame}}




{
\begin{frame}\frametitle{Identifiers}

	\begin{itemize}
	\item Rules for identifiers vary by language
			\item length, allowed characters, separators
			\item Both toekn kind and lexeme are significant
				\end{itemize}

\end{frame}}





{
\begin{frame}\frametitle{Comments}

	\begin{itemize}
	\item No syntatic meaning
			\item No token
			\item Scanner (Lexer) skips over comments
			\item Detect errors open comment, no close
				\end{itemize}

\end{frame}}






{
\begin{frame}\frametitle{Puntuation}

	\begin{itemize}
	\item Separators
			\item Typically single special characters
			\item Sometimes a short sequence.  i.e. comment operators
			\item Lexer returns longest match
			\item Only token kind (type) is significant
			\item May include location for error reporting
				\end{itemize}

\end{frame}}








\subsection{Variations}

{
\begin{frame}\frametitle{Free Form v Fixed Form}

	\begin{itemize}
	\item Free form

	\begin{itemize}
	\item White space doesn't matter
			\item Only token order
				\end{itemize}

			\item Fixed form

	\begin{itemize}
	\item Layout is critical
			\item primariy used in historical languages
				\end{itemize}

				\end{itemize}

\end{frame}}




{
\begin{frame}\frametitle{Ways to construct}

	\begin{itemize}
	\item Scanner generator - flex
			\item Hand written

	\begin{itemize}
	\item Performance

	\begin{itemize}
	\item Lexical Analysis can be bottleneck
			\item Minimize processing per character
			\item Large I/O reads
			\item We compile frequently so performance is important
				\end{itemize}

				\end{itemize}

				\end{itemize}

\end{frame}}




{
\begin{frame}\frametitle{Case Equivalance}

	\begin{itemize}
	\item Some languages are case insensitive

	\begin{itemize}
	\item Pascal, Ada
				\end{itemize}

			\item Some are not

	\begin{itemize}
	\item Java, C, C++
				\end{itemize}

				\end{itemize}

\end{frame}}





\section{Generators}
		
\subsection{Regular Expressions}

{
\begin{frame}\frametitle{Specifiying REs in Unix Tools}

	\begin{itemize}
	\item alternation [xy] x|y
			\item any character .
			\item Sequence abc
			\item Repition x*
			\item Optional x?y+
				\end{itemize}

\end{frame}}







\subsection{Using Flex}

{
\begin{frame}\frametitle{Help}

	\begin{itemize}
	\item http://flex.sourceforge.net/manual/index.html
				\end{itemize}

\end{frame}}



{
\begin{frame}\frametitle{File Structure}

	\begin{itemize}
	\item Definitions

	\begin{itemize}
	\item Sub-expressions like character
			\item include files for rules
				\end{itemize}

			\item Rules

	\begin{itemize}
	\item Regular Expression
			\item Action
			\item Precedence of expressions by order
				\end{itemize}

			\item Code

	\begin{itemize}
	\item Helper code
				\end{itemize}

				\end{itemize}

\end{frame}}






\section{Finite Automata}
		
\subsection{Definitions}

{
\begin{frame}\frametitle{Defined by alphabet and three operators}

	\begin{itemize}
	\item Precedence

	\begin{itemize}
	\item Kleene Closure (Repetition)
			\item Concatenation
			\item Alternation
				\end{itemize}

				\end{itemize}

\end{frame}}



{
\begin{frame}\frametitle{Regular Expressions = Finite Automata}

	\begin{itemize}
	\item Proof by construction to follow
				\end{itemize}

\end{frame}}



{
\begin{frame}\frametitle{Nondeterministic Finite Automata (NFA)}

	\begin{itemize}
	\item $\epsilon$ transitions

	\begin{itemize}
	\item Don't consume input
				\end{itemize}

			\item Equivalence

	\begin{itemize}
	\item NFA = FA
			\item Proof by construction to follow
				\end{itemize}

				\end{itemize}

\end{frame}}




\subsection{Limitations}

{
\begin{frame}\frametitle{Pumping Lemma}

\end{frame}}


\section{Scanner Generation}
		
\subsection{Overview}

{
\begin{frame}\frametitle{Automatic Generation}

	\begin{itemize}
	\item RE -> NFA
			\item NFA -> DFA
			\item DFA -> MFA
			\item MFA -> LA
				\end{itemize}

\end{frame}}






\subsection{Algorithims}

{
\begin{frame}\frametitle{Thompson's Construction}

	\begin{itemize}
	\item RE to NFA

	\begin{itemize}
	\item Base Case
			\item Alternation
			\item Concatenation
			\item Closure
				\end{itemize}

				\end{itemize}

\end{frame}}



{
\begin{frame}\frametitle{Evaluating Thompson's Construction}

	\begin{itemize}
	\item RE to NFA

	\begin{itemize}
	\item Base Case
			\item Alternation
			\item Concatenation
			\item Closure
				\end{itemize}

			\item Simple transformations for each operator

	\begin{itemize}
	\item Preserved Properties
			\item One start state
			\item One end state
			\item No transitions from end state
				\end{itemize}

				\end{itemize}

\end{frame}}




{
\begin{frame}\frametitle{Subset Construction}

	\begin{itemize}
	\item NFA to DFA
				\end{itemize}

\end{frame}}



{
\begin{frame}\frametitle{Hopcroft's Algorithim}

	\begin{itemize}
	\item DFA to Minimal DFA
			\item Find a set partion of states

	\begin{itemize}
	\item each set in state must have the same behavior on each input
				\end{itemize}

			\item start with start state in one partition and all others in a second
			\item find input for which states have different behavior
			\item continue
				\end{itemize}

\end{frame}}







\subsection{Scanner Construction}

{
\begin{frame}\frametitle{Recognizing multiple patterns}

	\begin{itemize}
	\item Multiple FA's
			\item Slightly different accept
			\item Back tracking
			\item Priority
				\end{itemize}

\end{frame}}






{
\begin{frame}\frametitle{Table Driven}

	\begin{itemize}
	\item One row per state
			\item One column per character
			\item Entries are next state
			\item Identical columns can be merged into a classifier
				\end{itemize}

\end{frame}}






{
\begin{frame}\frametitle{Direct Scanners}

	\begin{itemize}
	\item Replace each row (state) in table with code fragment
			\item transfer control to correct fragment based on input
			\item Violates structured programming, but it is generated
				\end{itemize}

\end{frame}}





{
\begin{frame}\frametitle{Hand-Code}

	\begin{itemize}
	\item Hand-written to optimize for I/O
				\end{itemize}

\end{frame}}



{
\begin{frame}\frametitle{Example: a(b|c)*}

\end{frame}}





\end{document}