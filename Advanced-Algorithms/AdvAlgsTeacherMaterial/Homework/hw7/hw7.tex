% latex first.tex
% latex first.tex
% xdvi first.dvi
% dvips -o first.ps first.dvi
% gv first.ps
% lpr first.ps
% pdflatex first.tex
% acroread first.pdf
% dvipdf first.dvi first.pdf
% xpdf first.pdf
\documentclass[11pt]{article}

\usepackage{latexsym}
%\newcommand{\epsfig}{\psfig}
%\usepackage{tabularx,booktabs,multirow,delarray,array}
%\usepackage{graphicx,amssymb,amsmath,amssymb,mathrsfs}
%\usepackage{hyperref}
\usepackage{graphicx}
\usepackage[linesnumbered, vlined, ruled]{algorithm2e}


%\usepackage[T1]{fontenc}


%\aboverulesep=0pt
%\belowrulesep=0pt

%\marginparwidth=0in
%\marginparsep=0in
\oddsidemargin=0.0in
\evensidemargin=0.0in
\headheight=0.0in
%\headsep=0in
\topmargin=-0.40in %0.35
\textheight=9.0in %9.1in
\textwidth=6.5in   %6.55in

\usepackage{fullpage}

%\setlength{\headheight}{0.2in}
%\setlength{\headsep}{0.2in}
%\setlength{\voffset}{-0.2in}

\def\report{{\em Report-Max}$(1)$}
\def\reportk{{\em Report-Max}$(k)$}

%\pagestyle{plain}

%\usepackage{listings}
%\lstloadlanguages{C, csh, make} \lstset{
%    language=C,tabsize=4,
%    keepspaces=true,
%    mathescape=true,
%    breakindent=22pt,
%    numbers=left,stepnumber=1,numberstyle=\footnotesize,
%    basicstyle=\normalsize,
%    showspaces=false,
%    flexiblecolumns=true,
%    breaklines=true, breakautoindent=true,breakindent=1em,
%    escapeinside={/*@}{@*/}
%}

%\lhead{Solution of Homework 1}
%\rhead{Haitao Wang}

\begin{document}
\baselineskip=14.0pt

\title{CS5050 \textsc{Advanced Algorithms}
\\{\Large Spring Semester, 2018}
\\ Assignment 7: Graph Algorithms II
\\ {\large {\bf Due Date: 3:00 p.m.}, Tuesday, Apr. 24, 2018 ({\bf at the beginning of CS5050 class})}}
\date{}
%\date{\today}


\maketitle
%\theoremstyle{plain}\newtheorem{theorem}{\textbf{Theorem}}

\vspace{-0.6in}

{\bf Note:} In this assignment, we assume all input graphs are represented by adjacency lists.

\begin{enumerate}

\item
{\bf (20 points)}
Given a directed graph $G$ of $n$ vertices and $m$ edges, each edge $(u,v)$ has a weight $w(u,v)$, which can be positive, zero, or negative.
The {\em bottleneck-weight} of any path in $G$ is defined to be the
{\bf largest} weight of all edges in the path. Let $s$ and $t$ be two vertices of $G$. A {\em minimum bottleneck-weight path} from $s$ to $t$ is a path with the smallest bottleneck-weight among all paths from $s$ to $t$ in $G$. Refer to Fig.~\ref{fig:bottleneck} for an example.


Modify Dijkstra's algorithm to compute a  minimum bottleneck-weight path from $s$ to $t$. Your algorithm should have the same time complexity as Dijkstra's algorithm.


\begin{figure}[h]
\begin{minipage}[t]{\linewidth}
\begin{center}
\includegraphics[totalheight=1.5in]{bottleneck.eps}
\caption{\footnotesize The following path is a minimum bottleneck-weight path from $s$ to $t$: $s,b,c,d,t$, whose bottleneck weight is $4$.}\label{fig:bottleneck}
\end{center}
\end{minipage}
\end{figure}





\item
Let $G=(V,E)$ be an undirected connected graph, and each edge $(u,v)$ has a positive weight $w(u,v)>0$. Let $s$ and $t$ be two vertices of $G$. Let $\pi(s,t)$ denote a shortest path from $s$ to $t$ in $G$. Let $T$ be a minimum spanning tree of $G$. Please answer the following questions and explain why you obtain your answers.

\begin{enumerate}
\item
Suppose we increase the weight of every edge of $G$ by a positive value $\delta>0$. Then, is $\pi(s,t)$ still a shortest path from $s$ to $t$?
{\hfill \bf (10 points)}



\item
Suppose we increase the weight of every edge of $G$ by a positive value $\delta>0$. Then, is $T$ still a minimum spanning tree of $G$?
{\hfill \bf (10 points)}

\end{enumerate}

{\bf Note:} For each of the two questions, your answer should be either ``Yes'' or ``Not necessary''. Again, please explain why you obtain your answers.

\item
{\bf (20 points)}
Let $G=(V,E)$ be an undirected connected graph of $n$ vertices and $m$ edges. Suppose each edge of $G$ has a color of either {\em blue} or {\em red}. Design an algorithm to
find a spanning tree $T$ of $G$ such that $T$ has as few red edges as possible. Your algorithm should run in $O((n+m)\log n)$ time. 



\end{enumerate}


{\bf Total Points: 60}
\end{document}

