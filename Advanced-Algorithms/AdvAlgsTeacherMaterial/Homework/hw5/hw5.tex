% latex first.tex
% latex first.tex
% xdvi first.dvi
% dvips -o first.ps first.dvi
% gv first.ps
% lpr first.ps
% pdflatex first.tex
% acroread first.pdf
% dvipdf first.dvi first.pdf
% xpdf first.pdf
\documentclass[11pt]{article}

\usepackage{latexsym}
%\newcommand{\epsfig}{\psfig}
%\usepackage{tabularx,booktabs,multirow,delarray,array}
%\usepackage{graphicx,amssymb,amsmath,amssymb,mathrsfs}
%\usepackage{hyperref}
\usepackage{graphicx}
\usepackage[linesnumbered, vlined, ruled]{algorithm2e}


%\usepackage[T1]{fontenc}


%\aboverulesep=0pt
%\belowrulesep=0pt

%\marginparwidth=0in
%\marginparsep=0in
\oddsidemargin=0.0in
\evensidemargin=0.0in
\headheight=0.0in
%\headsep=0in
\topmargin=-0.40in %0.35
\textheight=9.0in %9.1in
\textwidth=6.5in   %6.55in

\usepackage{fullpage}

%\setlength{\headheight}{0.2in}
%\setlength{\headsep}{0.2in}
%\setlength{\voffset}{-0.2in}

\def\report{{\em Report-Max}$(1)$}
\def\reportk{{\em Report-Max}$(k)$}

%\pagestyle{plain}

%\usepackage{listings}
%\lstloadlanguages{C, csh, make} \lstset{
%    language=C,tabsize=4,
%    keepspaces=true,
%    mathescape=true,
%    breakindent=22pt,
%    numbers=left,stepnumber=1,numberstyle=\footnotesize,
%    basicstyle=\normalsize,
%    showspaces=false,
%    flexiblecolumns=true,
%    breaklines=true, breakautoindent=true,breakindent=1em,
%    escapeinside={/*@}{@*/}
%}

%\lhead{Solution of Homework 1}
%\rhead{Haitao Wang}

\begin{document}
\baselineskip=14.0pt

\title{CS5050 \textsc{Advanced Algorithms}
\\{\Large Spring Semester, 2018}
\\ Assignment 5: Dynamic Programming
\\ {\large {\bf Due Date: 3:00 pm}, Tuesday, Apr. 3, 2018 ({\bf at the beginning of CS5050 class})}}
\date{}
%\date{\today}


\maketitle
%\theoremstyle{plain}\newtheorem{theorem}{\textbf{Theorem}}

\vspace{-0.6in}

\noindent
{\bf Note:} For each of the following problems, you will need to
design a dynamic programming algorithm. When you describe your algorithm,
please explain clearly the {\em subproblems} and the {\em dependency
relation} of your algorithm.

\begin{enumerate}

\item
{\bf (20 points)}
The knapsack problem we discussed in class is the following. Given
an integer $M$ and $n$ items of sizes $\{a_1, a_2, \ldots,
a_n\}$, determine whether there is a subset $S$ of the items such that the sum of the sizes of
all items in $S$ is exactly equal to $M$. We assume $M$ and all item sizes are positive integers.

Here we consider the following {\em unlimited version} of the problem. The
input is the same as before, except that there is an unlimited supply of each item. Specifically, we are given $n$ item sizes $a_1,a_2,\ldots,a_n$, which are positive integers.
The knapsack size is a positive integer $M$. The goal is to find a subset $S$ of items (to pack in the knapsack) such that the sum of the sizes of the items in $S$ is exactly $M$ and
each item is allowed to appear in $S$ multiple times.


For example, consider the following sizes of four items: \{2, 7, 9, 3\} and $M=14$. Here is a solution for the problem, i.e., use the first item once and use the fourth item four times, so the total sum of the sizes is $2+3\times 4=14$ (alternatively, you may also use the first item 4 times and the fourth item 2 times, i.e., $2\times 4 + 3\times 2=14$).

Design an $O(nM)$ time dynamic programming algorithm for solving this unlimited
knapsack problem. For simplicity, you only need to determine whether
there exists a solution (namely, if
there exists a solution, you do not need to report the actual solution subset).



\item
 {\bf (20 points)}
This is a problem from a student during his interview with Goldman Sachs
in Salt Lake City.

Given a set $A$ of $n$ positive integers $\{a_1, a_2, \ldots, a_n\}$ and
another positive integer $M$, find a subset of numbers of $A$ whose
sum is {\em closest} to $M$. In other words, find a subset $A'$ of $A$ such
that the absolute value $|M-\sum_{a\in A'}a|$ is minimized, where $\sum_{a\in
A'}a$ is the total sum of the numbers of $A'$. For the sake of simplicity, you only need to return the sum of the elements of the solution subset $A'$ without reporting the actual subset $A'$.

For example, suppose $A=\{1,4,7,12\}$ and $M=15$. Then, the solution subset is $A'=\{4,12\}$, and thus your algorithm only needs to return $4+12=16$ as the answer.

Let $K$ be the sum of all numbers of $A$.
Design a dynamic programming algorithm for the problem and your
algorithm should run in $O(nK)$ time (note that it is not $O(nM)$).


\item
 {\bf (20 points)}
Here is another variation of the knapsack problem. We are given $n$ items of sizes
$a_1,a_2,\ldots,a_n$, which are positive integers. Further, for each $1\leq i\leq n$, the $i$-th item $a_i$ has a positive value $value(a_i)$ (you may consider $value(a_i)$ as the amount of dollars the item is worth). The knapsack size is a positive integer $M$.

Now the goal is to find a subset $S$ of items such that the
sum of the sizes of all items in $S$ is {\bf at most} $M$ (i.e., $\sum_{a_i\in S}a_i\leq M$) and the
sum of the values of all items in $S$ is {\bf maximized} (i.e., $\sum_{a_i\in S}value(a_i)$ is maximized).

Design an $O(nM)$ time dynamic programming algorithm for the problem. For simplicity, you only need to report the sum of
the values of all items in the optimal solution subset $S$ and you do not need to report the
actual subset $S$.



\item
{ \bf (20 points)}
Given an array $A[1\ldots n]$ of $n$ distinct numbers (i.e., no two
numbers of $A$ are equal), design an
$O(n^2)$ time dynamic programming algorithm to find a {\em longest
monotonically increasing subsequence} of $A$. Your algorithm needs to
report not only the length but also the actual longest subsequence (i.e., report all elements in the subsequence).

Here is a formal definition of a {\em longest
monotonically increasing subsequence of $A$} (refer to the
example given below). First of all, a {\em subsequence} of $A$ is a subset of numbers of
$A$ such that if a number $a$ appears in front of another number
$b$ in the subsequence, then $a$ is also in front of $b$ in $A$.
Next, a subsequence of $A$ is {\em monotonically increasing} if for any two
numbers $a$ and $b$ such that $a$ appears
in front of $b$ in the subsequence, $a$ is smaller than $b$.
Finally, a {\em longest
monotonically increasing subsequence of $A$} refers to a monotonically
increasing subsequence of $A$ that is longest (i.e., has the maximum number of elements).

For example, if $A=\{20,5,14,8,10,3,12,7,16\}$, then a longest monotonically increasing
subsequence is $5,8,10,12,16$. Note that the answer may not be unique,
in which case you only need to report one such longest subsequence.


\end{enumerate}


{\bf Total Points: 80}
\end{document}

