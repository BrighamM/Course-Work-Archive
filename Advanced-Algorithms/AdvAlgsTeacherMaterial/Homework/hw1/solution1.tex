% latex first.tex
% latex first.tex
% xdvi first.dvi
% dvips -o first.ps first.dvi
% gv first.ps
% lpr first.ps
% pdflatex first.tex
% acroread first.pdf
% dvipdf first.dvi first.pdf
% xpdf first.pdf
\documentclass[11pt]{article}

\usepackage{latexsym}
%\newcommand{\epsfig}{\psfig}
\usepackage{tabularx,booktabs,multirow,delarray,array}
\usepackage{graphicx,amssymb,amsmath,amssymb,mathrsfs}
%\usepackage{hyperref}
\usepackage[linesnumbered, vlined, ruled]{algorithm2e}


%\usepackage[T1]{fontenc}


\aboverulesep=0pt
\belowrulesep=0pt

%\marginparwidth=0in
%\marginparsep=0in
\oddsidemargin=0.0in
\evensidemargin=0.0in
\headheight=0.0in
%\headsep=0in
\topmargin=-0.40in %0.35
\textheight=9.0in %9.1in
\textwidth=6.5in   %6.55in

%\usepackage{fullpage}

%\setlength{\headheight}{0.2in}
%\setlength{\headsep}{0.2in}
%\setlength{\voffset}{-0.2in}


%\pagestyle{plain}

%\usepackage{listings}
%\lstloadlanguages{C, csh, make} \lstset{
%    language=C,tabsize=4,
%    keepspaces=true,
%    mathescape=true,
%    breakindent=22pt,
%    numbers=left,stepnumber=1,numberstyle=\footnotesize,
%    basicstyle=\normalsize,
%    showspaces=false,
%    flexiblecolumns=true,
%    breaklines=true, breakautoindent=true,breakindent=1em,
%    escapeinside={/*@}{@*/}
%}

%\lhead{Solution of Homework 1}
%\rhead{Haitao Wang}

\begin{document}
\baselineskip=14.0pt

\title{CS5050 \textsc{Advanced Algorithms}
\\{\large Spring Semester, 2018}
\\ Homework Solution 1
\\ {\large Haitao Wang}}
\date{}
%\date{\today}


\maketitle
%\theoremstyle{plain}\newtheorem{theorem}{\textbf{Theorem}}

\vspace{-0.7in}

\begin{enumerate}

\item

\begin{enumerate}
\item
For $n=100$,
Sunway TaihuLight needs $\frac{2^{100}}{1.25\cdot 10^{17}}\approx 1.014\times 10^{13}$ seconds, which is about $\frac{2^{100}}{1.25\cdot 10^{17} \cdot 3600\cdot 24\cdot 365\cdot 100}\approx 3216$ centuries, to finish the algorithm.


\item
For $n=1000$,
Sunway TaihuLight needs $\frac{2^{1000}}{1.25\cdot 10^{17}}\approx 8.572\times 10^{283}$ seconds, which is about $\frac{2^{1000}}{1.25\cdot 10^{17} \cdot 3600\cdot 24\cdot 365\cdot 100}\approx 2.7\times 10^{274}$ centuries, to finish the algorithm.

\end{enumerate}



\item

Solution: $2^{500}$, $\log(\log n)^2$, $\log n$, $\log_4 n$ $\log^3 n$, $\sqrt{n}$, $2^{\log n}$, $n\log n$, $n^2\log^5 n$, $n^3$, $2^n$, $n!$. Note that $\log n = \Theta(\log_4 n)$, so you may put $\log_4 n$ in front of $\log n$ as well.


\item
For each of the following pairs of functions, indicate
whether it is one of the three cases: $f(n)=O(g(n))$, $f(n)=\Omega(g(n))$, or $f(n)=\Theta(g(n))$.

\begin{enumerate}
\item $f(n)=7\log n$ and $g(n)=\log n^3 + 56$.

Solution: $f(n)=\Theta(g(n))$.

Note: If your answer is $f(n)=O(g(n))$ or $f(n)=\Omega(g(n))$, then you will get three partial points.

\item $f(n)=n^2+n\log^3 n$ and $g(n)=6n^3+\log^2n$.

Solution: $f(n)=O(g(n))$.

\item $f(n)=5^n$ and $g(n)=n^22^n$.

Solution: $f(n)=\Omega(g(n))$. An easy way to see it is that $(\frac{5}{2})^n=\Omega(n^2)$.

\item $f(n)=n\log^2n$ and $g(n)=\frac{n^2}{\log^3 n}$.

Solution: $f(n)=O(g(n))$.

\item $f(n)=\sqrt{n}\log n$ and $g(n)=\log^8n+25$.

Solution: $f(n)=\Omega(g(n))$.

\item $f(n)= n\log n+6n$ and $g(n) = n\log_5 n-8n$

Solution: $f(n)=\Theta(g(n))$ because $\log_5 n=\frac{\log n}{\log_25}$.

Note: If your answer is $f(n)=O(g(n))$ or $f(n)=\Omega(g(n))$, then you will get three partial points.
\end{enumerate}





\item
	For each $i$ with $1\leq i\leq n$, denote by $x_i$ the $i$-th
	item. So the size of the item $x_i$  is $k_i$.
	
	Denote by $S$ the knapsack that we are going to pack and let $X$ be the total sum of the sizes 	 of the items in $S$. In the beginning of the algorithm, we have
	$S=\emptyset$ and $X=0$.

	We scan the items from $x_1$ to $x_n$ one by one. Consider the
	$i$-th item $x_i$ with $1\leq i\leq n$.

\begin{itemize}
\item

	If $X+k_i\leq K$, then we set $S=S\cup\{x_i\}$, i.e., we pack the item $x_i$ into $S$, and	 then update the value $X$ by setting $X=X+k_i$. We check whether $X\geq K/2$. If yes, the current knapsack $S$ is a factor of $2$ approximation solution and we terminate the algorithm (this also means that if the algorithm is not terminated yet, then $X<K/2$ always holds); otherwise, we proceed on the next item $x_{i+1}$.

\item
	If $X+k_i>K$, then the value
    $k_i$ must be larger than $K/2$ since $X<K/2$. We check whether
	$k_i\leq K$. If yes, we make our
	knapsack $S$ empty and let $S$ pack the only item $x_i$.
	Since $k_i> K/2$, the current knapsack $S$ is now a factor of $2$ approximation solution and we terminate the algorithm. Otherwise, we proceed on the next item $x_{i+1}$.
\end{itemize}

When the algorithm stops, the knapsack $S$ is a factor of $2$
approximation solution.

Since we consider each item at most once and we spend constant time on
considering each item, the running time of the algorithm is $O(n)$.
The following is the pseudocode.
	
%\end{description}

\begin{algorithm}[h]
	\caption{Finding a knapsack of factor $2$ approximation solution}
	\label{algo:case3}
	\SetAlgoNoLine
	\KwIn{The items $x_1,x_2,\ldots,x_n$ whose sizes are
	$k_1,k_2,\ldots,k_n$}
	\KwOut{A knapsack $S$ of factor $2$ approximation solution} \BlankLine
	%\tcc{All indices below are understood modulo $n$.}
	$S\leftarrow\emptyset$,
	$X\leftarrow 0$\;
	\For{$i\leftarrow 1$ \KwTo $n$}
	{
		\eIf{$X+k_i\leq K$}
		{
			$S\leftarrow S\cup\{x_i\}$\;
			$X\leftarrow X+k_i$\;
            \If{$X\geq K/2$}
			{
				break\tcc*[l]{terminate the algorithm}
			}
	    }
        (\tcc*[h]{$X+k_i>K$})
		{
			\If{$k_i\leq K$}
				{
					$S\leftarrow \emptyset$ \tcc*[l]{empty the knapsack}
					$S\leftarrow \{x_i\}$\;
%					$X=k_i$\;
					break\tcc*[l]{terminate the algorithm}
			     }
			}
	}
\end{algorithm}

\end{enumerate}

\end{document}


